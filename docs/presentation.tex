\documentclass{beamer}
\usepackage{hyperref}
\usepackage[orientation=landscape,size=custom,width=16,height=9,scale=0.5]{beamerposter} 
\usepackage{tikz}
\usepackage{xcolor}
\usetikzlibrary{automata}
\usepackage{listings}
\usepackage{graphicx}
\usepackage[english]{babel}
\usepackage[backend=biber,sorting=none]{biblatex}
\usepackage{syntax}
\renewcommand{\syntleft}{$\langle$\itshape}
\renewcommand{\syntright}{$\rangle$}
\setlength{\grammarindent}{6em}
\renewcommand{\ulitleft}{\ttfamily}
\renewcommand{\ulitright}{\normalfont}

\graphicspath{{images/}}
\DeclareTextFontCommand{\emph}{\bf}

\definecolor{mylinkcolor}{HTML}{6DDAC4}

\usetheme{esposito}

\lstset{
	% numbers=left,
	basicstyle=\ttfamily\scriptsize,
	breaklines=true,
	showstringspaces=false,
	postbreak=\mbox{$\hookrightarrow$\space},
	% numberstyle=\tiny,
	tabsize=4,
	backgroundcolor=\color[HTML]{f6f6f6},
	% xleftmargin=2em,
	% framexleftmargin=0.75em,
	commentstyle=\color[rgb]{0,0.6,0},
	keepspaces=true,
	keywordstyle=\color{blue},
	showspaces=false,
	showstringspaces=false,
	showtabs=false,
	stringstyle=\color{red},
	tabsize=2,
}
\lstset{
	inputpath={../src},
	language=haskell,
}

\definecolor{gradientstart}{HTML}{0F2027}
\definecolor{gradientend}{HTML}{2C5364}

\title{SIMPLI}
\subtitle{The Simple IMP Language Interpreter}
\author{Andrea Esposito}
\institute[]{University of Bari ``Aldo Moro''\\{\scriptsize Formal Methods in Computer Science}}
\date{Academic Year 2020--2021}

\addbibresource{bibliography.bib}

\begin{document}
{
	\settocustomtemplate
	\setbeamertemplate{footline}{%
        \begin{beamercolorbox}[wd=\paperwidth,ht=3.25ex,dp=2ex,center]{title in head/foot}%
			Released under CC BY-ND 4.0. Source code available under the GNU GPL v3.
        \end{beamercolorbox}%
	} 
	\begin{frame}
		\titlepage
	\end{frame}
	\addtocounter{framenumber}{-1}
}

\section{Introduction}
\begin{frame}
\frametitle{Introduction}
\begin{columns}
\column{0.75\textwidth}
\begin{itemize}
	\item SIMPLI is a simple interpreter for the IMP language
	\item Implemented in Haskell98
	\begin{itemize}
		\item Compiled through Cabal and GHC
		\item Tested using HUnit
	\end{itemize}
	\item Can be used as a:
	\begin{itemize}
		\item Library
		\item Executable
	\end{itemize}
	\item A few additions to the classic ``IMP'' language
	\begin{itemize}
		\item Arrays
		\item Strings
		\item Comments
	\end{itemize}
\end{itemize}
\column{0.25\textwidth}
\includegraphics[width=\textwidth]{haskell}
\end{columns}
\end{frame}

\section{The Language Grammar}
\begin{frame}[fragile]
\footnotesize
\frametitle{The Language Grammar (1/5)}
\begin{columns}
\column{0.45\textwidth}
\begin{block}{Integers and Numbers}
\begin{grammar}
	<integer> ::= "-" <natural> | <natural>

	<natural> ::= <digit> | <digit> <natural>

	<digit> ::= "0" | "1" | "2" | "3" | "4"
	\alt "5" | "6" | "7" | "8" | "9"
\end{grammar}
\end{block}
\column{0.45\textwidth}
\begin{block}{Identifiers}
\begin{grammar}
	<identifier> ::= <upper> <alphanum>
	\alt <lower> <alphanum>
	\alt <upper> | <lower> 
	
	<alphanum> ::= <upper> <alphanum>
	\alt <lower> <alphanum>
	\alt <natural> <alphanum>
	\alt <upper> | <lower>
	\alt <natural>

	<lower> ::= "a" | "b" | ... | "z"

	<upper> ::= "A" | "B" | ... | "Z"
\end{grammar}
\end{block}
\end{columns}
\end{frame}

\begin{frame}[fragile]
	\footnotesize
	\frametitle{The Language Grammar (2/5)}
	\begin{columns}
		\column{0.45\textwidth}
		\begin{block}{Aexp}
			\begin{grammar}
				<aexp> ::= <aterm>
				\alt <aterm> "+" <aexp>
				\alt <aterm> "-" <aexp>

				<aterm> ::= <afactor> | <afactor> "*" <aterm>

				<afactor> ::= "(" <aexp> ")"
				\alt `|' <array> `|'
				\alt `|' <cexp> `|'
				\alt <identifier> "[" <aexp> "]"
				\alt <integer> | <identifier>
			\end{grammar}
		\end{block}
		\column{0.45\textwidth}
		\begin{block}{Bexp}
			\begin{grammar}
				<bexp> ::= <bterm> "or" <bexp> | <bterm>

				<bterm> ::= <bfactor> "and" <bterm> | <bfactor>

				<bfactor> ::= "true" | "false"
				\alt "!" <bfactor>
				\alt "(" <bexp> ")"
				\alt <bcomparison>

				<bcomparison> ::= <aexp> "=" <aexp>
				\alt <aexp> "<=" <aexp>
				\alt <aexp> "<" <aexp>
				\alt <cexp> "=" <cexp>
			\end{grammar}
		\end{block}
	\end{columns}
\end{frame}

\begin{frame}[fragile]
	\footnotesize
	\frametitle{The Language Grammar (3/5)}
	\begin{columns}
		\column{0.36\textwidth}
		\begin{grammar}
			<array> ::= <basicarray>
			\alt <basicarray> "++" <array>

			<basicarray> ::= "[" <asequence> "]"
			\alt <identifier>

			<asequence> ::= <aexp>
			\alt <aexp> "," <asequence>
		\end{grammar}

		\column{0.63\textwidth}
		\begin{itemize}
			\item A new data type: arrays
			\item New operations defined
				\begin{itemize}
					\footnotesize
					\item \textbf{Concatenation:} given $A=\left[ a_1,\ldots a_n\right]$
						and $B = \left[ b_1,\ldots, b_m\right]$,
						$A++B=\left[a_1,\ldots, a_n, b_1, \ldots, b_m\right]$
					\item \textbf{Indexing:} given $A=\left[ a_1,\ldots a_n\right]$,
						$\forall i = 0,\ldots, n-1: A[i] = a_{i+1}$
					\item \textbf{Size operator:} given $A=\left[ a_1,\ldots a_n\right]$,
						$|A|=n$.
				\end{itemize}
			\item Arrays are just variables
				\begin{itemize}
					\footnotesize
					\item Parsed as part of $\mathrm{Aexp}$
					\item Used in assignments
				\end{itemize}
		\end{itemize}
	\end{columns}
\end{frame}

\begin{frame}[fragile]
	\footnotesize
	\frametitle{The Language Grammar (4/5)}
	\begin{columns}
		\column{0.36\textwidth}
		\begin{grammar}
			<cexp> ::= <cterm> ++ <cexp> | <cterm>

			<cterm> ::= "\"" <string> "\"" | <identifier>

			<string> ::= "[^\"]" <string> | "[^\"]"
		\end{grammar}

		\column{0.63\textwidth}
		\begin{itemize}
			\item A new data type: strings
			\item New operations defined
				\begin{itemize}
					\footnotesize
					\item \textbf{Concatenation:} given $S_1=\texttt{"}
						a_1,\ldots a_n\texttt{"}$ and $S_2 = \texttt{"}
						b_1\ldots b_m\texttt{"}$, $S_1++S_2=\texttt{"}a_1\ldots
						a_n b_1 \ldots b_m\texttt{"}$
					\item \textbf{Indexing:} given $S_1=\texttt{"} a_1\ldots
						a_n\texttt{"}$, $\forall i = 0,\ldots, n-1: S_1[i] =
						a_{i+1}$
					\item \textbf{Size operator:} given $S_1=\texttt{"}
						a_1\ldots a_n\texttt{"}$, $|S_1|=n$.
				\end{itemize}
			\item Strings are similar to arrays
				\begin{itemize}
					\footnotesize
					\item Parsed as part of $\mathrm{Aexp}$
					\item Used in assignments
				\end{itemize}
			\item Strings are immutable!
			\item Note: \texttt{[\^{}"]} means ``all characters except double
				quotes''
		\end{itemize}
	\end{columns}
\end{frame}

\begin{frame}[fragile]
\footnotesize
\frametitle{The Language Grammar (5/5)}
\begin{block}{Com}
\begin{columns}
\column{0.40\textwidth}
\begin{grammar}
	<program> ::= <command>
	\alt <command> ";"
	\alt <command> ";" <program>

	<command> ::= <assignment>
	\alt <ifThenElse>
	\alt <while> | "skip"

\end{grammar}
\column{0.60\textwidth}
\begin{grammar}
	<ifThenElse> ::= "if" <bexp> "then" <program> "end"
	\alt "if" <bexp> "then" <program> "else" <program> "end"

	<while> ::= "while" <bexp> "do" <program> "end"

	<assignment> ::= <identifier> ":=" <aexp>
	\alt <identifier> ":=" <array>
	\alt <identifier> ":=" <cexp>
	\alt <identifier> "[" <aexp> "]" ":=" <aexp>
\end{grammar}
\end{columns}
\end{block}
\end{frame}

\section{The Strategy}
\begin{frame}
\frametitle{Strategy}
\begin{itemize}
	\item Eager evaluation
	\item Short-circuited evaluation of boolean expressions
	\item Environment management
		\begin{itemize}
			\item The environment is the set of variables used in the
				computation
			\item All computations start in a ``sandbox'', i.e. an empty
				environment
		\end{itemize}
	\item A few additions:
		\begin{itemize}
			\item The parser ignores unnecessary white space
			\item The parser recognizes comments that starts with an hash
				symbol ``\texttt{\#}''
		\end{itemize}
\end{itemize}
\end{frame}

\section{The Environment}
\begin{frame}
	\frametitle{The Environment}
	\begin{columns}
		\column{0.45\textwidth}
		\small
		\begin{itemize}
			\item Variables have a name, a type (\texttt{int}) and a value
			\item An environment is a set of variables
			\item An environment can be updated by adding or modifying a
				variable
			\item A variable can be read from an environment using its name
		\end{itemize}
		\column{0.45\textwidth}
		\lstinputlisting[firstline=11, lastline=15]{Environment.hs}
		\lstinputlisting[firstline=23, lastline=23]{Environment.hs}
	\end{columns}
\end{frame}

\begin{frame}
	\frametitle{The Environment}
	\small
	\centerline{Functions to deal with variables and arrays are provided}
	\begin{columns}
		\column{0.4\textwidth}
		\lstinputlisting[basicstyle=\ttfamily\tiny,firstline=25, lastline=29]{Environment.hs}
		\lstinputlisting[basicstyle=\ttfamily\tiny,firstline=31, lastline=35]{Environment.hs}
		\column{0.6\textwidth}
		\lstinputlisting[basicstyle=\ttfamily\tiny,firstline=37, lastline=42]{Environment.hs}
		\lstinputlisting[basicstyle=\ttfamily\tiny,firstline=44, lastline=52]{Environment.hs}
	\end{columns}
\end{frame}

\section{Implementation}
\begin{frame}
	\frametitle{Implementation}
	\begin{itemize}
		\item Each of the three main parsers (for $\mathrm{Aexp}$, $\mathrm{Bexp}$ and $\mathrm{Com}$) are
			implemented as the Haskell representation of the generative grammar
			\begin{itemize}
				\item For each parser, it exist an executing and a
					non-executing version. The non executing version is denoted
					with the prefix ``\texttt{R.}''.
			\end{itemize}
		\item Basic parsers are needed
		\begin{itemize}
			\item Fundamentals: \texttt{symbol}, \texttt{item},
				\texttt{satisfies}, \texttt{notsymbol}
			\item Based on a grammar: \texttt{identifier}, \texttt{integer}
		\end{itemize}
		\item A way of dealing with the environment while inside of a parser is needed
	\end{itemize}
\end{frame}

\begin{frame}
	\frametitle{The Parser Monad}
	\begin{center}
		\lstinline{newtype Parser a = P(Env -> String -> Maybe (Env, a, String))}
	\end{center}
	\begin{columns}
		\column{0.3\textwidth}
		\footnotesize
		\begin{itemize}
			\item A new ``Parser'' type is defined
				\begin{itemize}
					\footnotesize
					\item Using the monad \lstinline|Maybe| to manage errors
				\end{itemize}
			\item A parser is a Functor, meaning we can apply a function
				\texttt{fmap}
			\item A parser is an Applicative, meaning it can be applied
				sequentially (discarding the previously computed results)
		\end{itemize}
		\column{0.65\textwidth}
		\lstinputlisting[firstline=10,lastline=14]{Parser/Core.hs}
		\lstinputlisting[firstline=16, lastline=23]{Parser/Core.hs}
	\end{columns}
\end{frame}

\begin{frame}
	\frametitle{The Parser Monad}
	\begin{columns}
		\column{0.3\textwidth}
		\footnotesize
		\begin{itemize}
			\item A parser is a Monad, meaning we can sequentially apply
				different parsers
			\item A parser can also be defined as an \texttt{Alternative}, that
				allows to use the syntax \lstinline{a <|> b} to express the
				fact that the parser \texttt{b} should be applied if \texttt{a}
				fails.
		\end{itemize}
		\column{0.65\textwidth}
		\lstinputlisting[firstline=25, lastline=32]{Parser/Core.hs}
		\lstinputlisting[firstline=34, lastline=41]{Parser/Core.hs}
	\end{columns}
\end{frame}

\begin{frame}
	\frametitle{Environment and Parsers}
	\begin{itemize}
		\footnotesize
		\item Functions to deal with the environment while being in a parser,
			without loosing the possibility of using ``do-blocks'' to improve
			readability
		\item Moving the environment functions in the Parser monad
			\begin{itemize}
				\item \lstinline{modifyEnv} $\mapsto$ \lstinline{updateEnv},
					\lstinline{modifyArray} $\mapsto$ \lstinline{saveArray}
				\item \lstinline{searchVariable} $\mapsto$ \lstinline{readVariable},
					\lstinline{searchArray} $\mapsto$ \lstinline{readArray}
			\end{itemize}
	\end{itemize}
	\lstinputlisting[firstline=13, lastline=14]{Parser/Environment.hs}
	\lstinputlisting[firstline=16, lastline=19]{Parser/Environment.hs}
	\lstinputlisting[firstline=21, lastline=22]{Parser/Environment.hs}
	\lstinputlisting[firstline=24, lastline=27]{Parser/Environment.hs}
\end{frame}

\begin{frame}
	\frametitle{Arithmetic Parsing (\texttt{aexp})}
	\begin{columns}
		\column{0.4\textwidth}
		\lstinputlisting[basicstyle=\ttfamily\tiny,firstline=8,lastline=17]{Parser/Aexp.hs}
		\lstinputlisting[basicstyle=\ttfamily\tiny,firstline=19,lastline=24]{Parser/Aexp.hs}
		\column{0.6\textwidth}
		\lstinputlisting[basicstyle=\ttfamily\tiny,firstline=26,lastline=52]{Parser/Aexp.hs}
	\end{columns}
\end{frame}

\begin{frame}
	\frametitle{Arithmetic Parsing: Arrays (\texttt{array})}
	\begin{columns}
		\column{0.4\textwidth}
		\lstinputlisting[firstline=56,lastline=61]{Parser/Aexp.hs}
		\column{0.4\textwidth}
		\lstinputlisting[firstline=63,lastline=69]{Parser/Aexp.hs}
		\lstinputlisting[firstline=71,lastline=77]{Parser/Aexp.hs}
	\end{columns}
\end{frame}

\begin{frame}
	\frametitle{Arithmetic Parsing: Strings (\texttt{cexp})}
	\begin{columns}
		\column{0.4\textwidth}
		\lstinputlisting[firstline=81,lastline=91]{Parser/Aexp.hs}
		\column{0.5\textwidth}
		\lstinputlisting[firstline=93,lastline=102]{Parser/Aexp.hs}
		\lstinputlisting[firstline=104,lastline=109]{Parser/Aexp.hs}
	\end{columns}
\end{frame}

\begin{frame}[fragile]
	\frametitle{Boolean Parsing (\texttt{bexp})}
	\begin{columns}
		\column{0.4\textwidth}
		\lstinputlisting[basicstyle=\ttfamily\tiny,firstline=8,lastline=16]{Parser/Bexp.hs}
		\lstinputlisting[basicstyle=\ttfamily\tiny,firstline=28,lastline=40]{Parser/Bexp.hs}
		\column{0.4\textwidth}
		\lstinputlisting[basicstyle=\ttfamily\tiny,firstline=18,lastline=26]{Parser/Bexp.hs}
		\lstinputlisting[basicstyle=\ttfamily\tiny,firstline=45]{Parser/Bexp.hs}
	\end{columns}
	\centerline{\scriptsize Note: short circuited evaluation implemented in \texttt{bexp} and \texttt{bterm}}
\end{frame}

\begin{frame}[fragile]
	\frametitle{Command Parsing (\texttt{program}, 1/3)}
	\begin{columns}
		\column{0.45\textwidth}
		\lstinputlisting[basicstyle=\ttfamily\tiny,firstline=13,lastline=19]{Parser/Com.hs}
\begin{lstlisting}[basicstyle=\ttfamily\tiny]
command :: Parser String
command = assignment <|> ifThenElse
          <|> while <|> symbol "skip"
\end{lstlisting}
		\column{0.45\textwidth}
		\lstinputlisting[basicstyle=\ttfamily\tiny,firstline=53,lastline=68]{Parser/Com.hs}
	\end{columns}
\end{frame}

\begin{frame}
	\frametitle{Command Parsing (arrays and strings, 2/3)}
	\centerline{\small Arrays and strings are treated during assignments}
	\lstinputlisting[basicstyle=\ttfamily\tiny,firstline=25,lastline=50]{Parser/Com.hs}
\end{frame}

\begin{frame}[fragile]
	\frametitle{Command Parsing (\texttt{program}, 3/3)}
	\begin{columns}
		\column{0.45\textwidth}
		\small
		A special note should be given on the while-do construct
		\begin{itemize}
			\small
			\item To deal with iteration, the while construct’s code is read
				into a variable and is repeated after the “end” keyword if the
				boolean condition is true.
			\begin{itemize}
				\footnotesize
				\item Fixed point operator
				\item Function \texttt{repeatWhile}
			\end{itemize}
		\end{itemize}
		\column{0.45\textwidth}
		\lstinputlisting[basicstyle=\ttfamily\tiny,firstline=70,lastline=83]{Parser/Com.hs}
\begin{lstlisting}[basicstyle=\ttfamily\tiny]
repeatWhile :: String -> Parser String
repeatWhile s = P(\env input ->
                  Just (env, "", s ++ input))
\end{lstlisting}
	\end{columns}
\end{frame}


\begin{frame}[fragile]
	\frametitle{The Actual Parsing and Execution}
	\begin{columns}
		\column{0.35\textwidth}
		\small
		A single function is actually exposed
		\begin{itemize}
			\small
		\item eval executes a program in an empty environment
			\begin{itemize}
				\footnotesize
				\item All comments and white space (unless required) are
					stripped from the input program
				\item Functions \texttt{removeComments} and
					\texttt{removeWhitespace}
			\end{itemize}
		\end{itemize}
		\column{0.6\textwidth}
		\lstinputlisting[basicstyle=\ttfamily\tiny,firstline=32]{Parser.hs}
	\end{columns}
\end{frame}

\begin{frame}[fragile]
	\frametitle{Comments and White Space}
	\begin{itemize}
		\footnotesize
		\item Comments are removed on a line-by-line fashion: after an hash
			symbol, the rest of the line is a comment
	\end{itemize}
	\lstinputlisting[basicstyle=\ttfamily\tiny,firstline=8,lastline=15]{Parser.hs}
	\begin{itemize}
		\footnotesize
		\item Some keywords (like \texttt{if} and \texttt{while}) need
			white space, but it's easier to manage these as exceptions by
			removing all spaces and readding them where needed.
	\end{itemize}
	\lstinputlisting[basicstyle=\ttfamily\tiny,firstline=17,lastline=30]{Parser.hs}
\end{frame}

\begin{frame}
	\frametitle{The Comments Syntax}
	\begin{figure}[H]
		\centering
		\begin{tikzpicture}[node distance=3cm,->]
			\node[state, initial] (q0) {$q_0$};
		\node[state, accepting, right of=q0] (q1) {$q_1$};

		\draw (q0) edge[loop above] node{$\ast - \texttt{"a"}$} (q0)
		(q0) edge[above] node{\texttt{\#}} (q1)
		(q1) edge[loop above] node{$\ast$} (q1);
		\end{tikzpicture}
		\caption[Finite state automaton to recognize comments]{Finite state automaton to recognize comments. The $\ast$ symbol denotes all characters. The notation $\ast - \texttt{"a"}$ denotes all characters except ``a''.}
		\label{fig:comment-automaton}
	\end{figure}
\end{frame}

\section{Usage}
\begin{frame}[fragile]
	\frametitle{Usage}
	\begin{columns}
		\column{0.5\textwidth}
		\begin{block}{As a library}
\begin{lstlisting}
$ cd /path/to/src
$ hugs
Hugs> :load Parser
Parser> eval "x := 3"
[x = 3]
\end{lstlisting}
			\end{block}
		\column{0.5\textwidth}
		\begin{block}{As an executable}
\begin{lstlisting}[language=sh]
$ cat file.imp
x := 3;
$ simpli file.imp
[x = 3]
\end{lstlisting}
\begin{lstlisting}[language=sh]
$ echo "x := 3" | simpli
[x = 3]
\end{lstlisting}
\begin{lstlisting}[language=sh]
$ simpli -c "x := 3"
[x = 3]
\end{lstlisting}
		\end{block}
	\end{columns}
\end{frame}

% \section{Bibliography}
% \begin{frame}
	% \frametitle{References}
	% \nocite{Hutton2017}
	% \printbibliography
% \end{frame}

{
	\settocustomtemplate
	\setbeamertemplate{footline}{} 
	\begin{frame}
		\color{white}
		\vfill
		\begin{center}
			\Large
			The End
		\end{center}
		\begin{center}
			\small
			Questions?
		\end{center}
		\vfill
		\begin{center}
			\scriptsize
			Source code:
			\href{https://github.com/espositoandrea/simpli}{%
				\color{mylinkcolor}github.com/espositoandrea/simpli
			}
		\end{center}
	\end{frame}
}
\end{document}
