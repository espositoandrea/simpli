\documentclass{beamer}
\usepackage{hyperref}
\usepackage[orientation=landscape,size=custom,width=16,height=9,scale=0.5]{beamerposter} 
\usepackage{tikz}
\usepackage{xcolor}
\usepackage{listings}
\usepackage{graphicx}
\usepackage[english]{babel}
\usepackage{syntax}
\renewcommand{\syntleft}{$\langle$\itshape}
\renewcommand{\syntright}{$\rangle$}
\setlength{\grammarindent}{6em}
\renewcommand{\ulitleft}{\ttfamily}
\renewcommand{\ulitright}{\normalfont}

\graphicspath{{images/}}
\DeclareTextFontCommand{\emph}{\bf}

\definecolor{mylinkcolor}{HTML}{6DDAC4}

\usetheme{esposito}

\lstset{
	% numbers=left,
	basicstyle=\ttfamily\scriptsize,
	breaklines=true,
	showstringspaces=false,
	postbreak=\mbox{$\hookrightarrow$\space},
	% numberstyle=\tiny,
	tabsize=4,
	backgroundcolor=\color[HTML]{f6f6f6},
	% xleftmargin=2em,
	% framexleftmargin=0.75em,
	commentstyle=\color[rgb]{0,0.6,0},
	keepspaces=true,
	keywordstyle=\color{blue},
	showspaces=false,
	showstringspaces=false,
	showtabs=false,
	stringstyle=\color{red},
	tabsize=2,
}
\lstset{
	inputpath={../src},
	language=haskell,
}


\definecolor{gradientstart}{HTML}{0F2027}
\definecolor{gradientend}{HTML}{2C5364}

\title{SIMPLI}
\subtitle{The Simple IMP Language Interpreter}
\author{Andrea Esposito}
\institute[]{University of Bari ``Aldo Moro''\\{\scriptsize Formal Methods in Computer Science}}
\date{\today}



\begin{document}
{
	\settocustomtemplate
	\setbeamertemplate{footline}{} 
	\begin{frame}
		\titlepage
		\vfill
		\centerline{%
			\color{white}\tiny
			Released under CC BY-ND 4.0. Source code available under the GNU
			GPL v3.
		}
	\end{frame}
	\addtocounter{framenumber}{-1}
}

\section{Introduction}
\begin{frame}
\frametitle{Introduction}
\begin{columns}
\column{0.75\textwidth}
\begin{itemize}
	\item SIMPLI is a simple interpreter for the IMP language
	\item Implemented in Haskell98
	\begin{itemize}
		\item Compiled through Cabal and GHC
		\item Tested using HUnit
	\end{itemize}
	\item Can be used as a:
	\begin{itemize}
		\item Library
		\item Executable
	\end{itemize}
\end{itemize}
\column{0.25\textwidth}
\includegraphics[width=\textwidth]{haskell}
\end{columns}
\end{frame}

\section{The Language Grammar}
\begin{frame}[fragile]
\footnotesize
\frametitle{The Language Grammar (1/3)}
\begin{columns}
\column{0.45\textwidth}
\begin{block}{Integers and Numbers}
\begin{grammar}
	<integer> ::= "-" <natural> | <natural>

	<natural> ::= <digit> | <digit> <natural>

	<digit> ::= "0" | "1" | "2" | "3" | "4"
	\alt "5" | "6" | "7" | "8" | "9"
\end{grammar}
\end{block}
\column{0.45\textwidth}
\begin{block}{Identifiers}
\begin{grammar}
	<identifier> ::= <upper> <alphanum>
	\alt <lower> <alphanum>
	\alt <upper> | <lower> 
	
	<alphanum> ::= <upper> <alphanum>
	\alt <lower> <alphanum>
	\alt <natural> <alphanum>
	\alt <upper> | <lower>
	\alt <natural>

	<lower> ::= "a" | "b" | ... | "z"

	<upper> ::= "A" | "B" | ... | "Z"
\end{grammar}
\end{block}
\end{columns}
\end{frame}
\begin{frame}[fragile]
\footnotesize
\frametitle{The Language Grammar (2/3)}
\begin{columns}
\column{0.45\textwidth}
\begin{block}{Aexp}
\begin{grammar}
	<aexp> ::= <aterm>
	\alt <aterm> "+" <aexp>
	\alt <aterm> "-" <aexp>

	<aterm> ::= <afactor>
	\alt <afactor> "*" <aterm>

	<afactor> ::= "(" <aexp> ")"
	\alt <integer>
	\alt <identifier>
\end{grammar}
\end{block}
\column{0.45\textwidth}
\begin{block}{Bexp}
\begin{grammar}
	<bexp> ::= <bterm>
	\alt <bterm> "or" <bexp>

	<bterm> ::= <bfactor>
	\alt <bfactor> "and" <bterm>

	<bfactor> ::= "true" | "false"
	\alt "!" <bfactor>
	\alt "(" <bexp> ")"
	\alt <bcomparison>

	<bcomparison> ::= <aexp> "=" <aexp>
	\alt <aexp> "<=" <aexp>
\end{grammar}
\end{block}
\end{columns}
\end{frame}
\begin{frame}[fragile]
\footnotesize
\frametitle{The Language Grammar (3/3)}
\begin{block}{Com}
\begin{columns}
\column{0.40\textwidth}
\begin{grammar}
	<program> ::= <command>
	\alt <command> ";"
	\alt <command> ";" <program>

	<command> ::= <assignment>
	\alt <ifThenElse>
	\alt <while> | "skip"

	<assignment> ::= <identifier> ":=" <aexp>

\end{grammar}
\column{0.60\textwidth}
\begin{grammar}
	<ifThenElse> ::= "if" <bexp> "then" <program> "end"
	\alt "if" <bexp> "then" <program> "else" <program> "end"

	<while> ::= "while" <bexp> "do" <program> "end"
\end{grammar}
\end{columns}
\end{block}
\end{frame}

\section{The Strategy}
\begin{frame}
\frametitle{Strategy}
\begin{itemize}
	\item Eager evaluation
	\item Short-circuited evaluation of boolean expressions
	\item Environment management
		\begin{itemize}
			\item The environment is the set of variables used in the
				computation
			\item All computations start in a ``sandbox'', i.e. an empty
				environment
		\end{itemize}
	\item A few additions:
		\begin{itemize}
			\item The parser ignores unnecessary white space
			\item The parser recognizes comments that starts with an hash
				symbol ``\texttt{\#}''
		\end{itemize}
\end{itemize}
\end{frame}

\section{The Environment}
\begin{frame}
	\frametitle{The Environment}
	\begin{columns}
		\column{0.35\textwidth}
		\small
		\begin{itemize}
			\item Variables have a name, a type (\texttt{int}) and a value
			\item An environment is a set of variables
			\item An environment can be updated by adding or modifying a
				variable
			\item A variable can be read from an environment using its name
		\end{itemize}
		\column{0.6\textwidth}
		\lstinputlisting[firstline=3,lastline=6]{Environment.hs}
		\lstinputlisting[firstline=12,lastline=12]{Environment.hs}
		\lstinputlisting[firstline=14,lastline=18]{Environment.hs}
		\lstinputlisting[firstline=20,lastline=24]{Environment.hs}
	\end{columns}
\end{frame}

\section{Implementation}
\begin{frame}
	\frametitle{Implementation}
	\begin{itemize}
		\item Each of the three main parsers (for $\mathrm{Aexp}$, $\mathrm{Bexp}$ and $\mathrm{Com}$) are
			implemented as the Haskell representation of the generative grammar
			\begin{itemize}
				\item For each parser, it exist an executing and a
					non-executing version. The non executing version is denoted
					with the prefix ``\texttt{R.}''.
			\end{itemize}
		\item Basic parsers are needed
		\begin{itemize}
			\item Fundamentals: \texttt{symbol}
			\item Based on a grammar: \texttt{identifier}, \texttt{integer}
		\end{itemize}
		\item A way of dealing with the environment while inside of a parser is needed
	\end{itemize}
\end{frame}

\begin{frame}
	\frametitle{The Parser Monad}
	\begin{center}
		\lstinline{newtype Parser a = P(Env -> String -> Maybe (Env, a, String))}
	\end{center}
	\begin{columns}
		\column{0.3\textwidth}
		\footnotesize
		\begin{itemize}
			\item A new ``Parser'' type is defined
				\begin{itemize}
					\footnotesize
					\item Using the monad \lstinline|Maybe| to manage errors
				\end{itemize}
			\item A parser is a Functor, meaning we can apply a function
				\texttt{fmap}
			\item A parser is a Monad, meaning we can sequentially apply
				different parsers
		\end{itemize}
		\column{0.65\textwidth}
		\lstinputlisting[firstline=10,lastline=14]{Parser/Core.hs}
		\lstinputlisting[firstline=25, lastline=32]{Parser/Core.hs}
	\end{columns}
\end{frame}

\begin{frame}
	\frametitle{The Parser Monad}
	\begin{columns}
		\column{0.3\textwidth}
		\footnotesize
		\begin{itemize}
			\item Being a Monad, a parser must also be an Applicative
			\item A parser can also be defined as an \texttt{Alternative}, that
				allows to use the syntax \lstinline{a <|> b} to express the
				fact that the parser \texttt{b} should be applied if \texttt{a}
				fails.
		\end{itemize}
		\column{0.65\textwidth}
		\lstinputlisting[firstline=16, lastline=23]{Parser/Core.hs}
		\lstinputlisting[firstline=34, lastline=41]{Parser/Core.hs}
	\end{columns}
\end{frame}

\begin{frame}
	\frametitle{Environment and Parsers}
	\begin{itemize}
		\footnotesize
		\item Functions to deal with the environment while being in a parser,
			without loosing the possibility of using ``do-blocks'' to improve
			readability
		\item Moving the environment functions in the Parser monad
			\begin{itemize}
				\item \lstinline{modifyEnv} $\mapsto$ \lstinline{updateEnv}
				\item \lstinline{searchVariable} $\mapsto$ \lstinline{readVariable}
			\end{itemize}
	\end{itemize}
	\lstinputlisting[firstline=5,lastline=6]{Parser/Environment.hs}
	\lstinputlisting[firstline=8,lastline=11]{Parser/Environment.hs}
\end{frame}

\begin{frame}
	\frametitle{Arithmetic Parsing (\texttt{aexp})}
	\begin{columns}
		\column{0.4\textwidth}
		\lstinputlisting[firstline=8,lastline=18]{Parser/Aexp.hs}
		\column{0.4\textwidth}
		\lstinputlisting[firstline=20,lastline=25]{Parser/Aexp.hs}
		\lstinputlisting[firstline=27,lastline=35]{Parser/Aexp.hs}
	\end{columns}
\end{frame}

\begin{frame}[fragile]
	\frametitle{Boolean Parsing (\texttt{bexp})}
	\begin{columns}
		\column{0.4\textwidth}
		\lstinputlisting[basicstyle=\ttfamily\tiny,firstline=8,lastline=16]{Parser/Bexp.hs}
\begin{lstlisting}[basicstyle=\ttfamily\tiny]
bfactor :: Parser Bool
bfactor = (do symbol "!"
              b <- bfactor
              return $ not b)
          <|> (do symbol "("
                  b <- bexp
                  symbol ")"
                  return b)
          <|> (do symbol "true"
                  return True)
          <|> (do symbol "false"
                  return False)
          <|> bcomparison
\end{lstlisting}
		\column{0.4\textwidth}
		\lstinputlisting[basicstyle=\ttfamily\tiny,firstline=18,lastline=26]{Parser/Bexp.hs}
		\lstinputlisting[basicstyle=\ttfamily\tiny,firstline=45,lastline=54]{Parser/Bexp.hs}
	\end{columns}
	\centerline{\scriptsize Note: short circuited evaluation implemented in \texttt{bexp} and \texttt{bterm}}
\end{frame}

\begin{frame}[fragile]
	\frametitle{Command Parsing (\texttt{program}, 1/2)}
	\begin{columns}
		\column{0.45\textwidth}
		\lstinputlisting[basicstyle=\ttfamily\tiny,firstline=13,lastline=19]{Parser/Com.hs}
\begin{lstlisting}[basicstyle=\ttfamily\tiny]
command :: Parser String
command = assignment <|> ifThenElse
          <|> while <|> (symbol "skip")
\end{lstlisting}
\begin{lstlisting}[basicstyle=\ttfamily\tiny]
assignment :: Parser String
assignment = do var <- identifier
                symbol ":="
                val <- aexp
                updateEnv Variable{ name=var
                                  , vtype="int"
                                  , value=val
                                  }
\end{lstlisting}
		\column{0.45\textwidth}
		\lstinputlisting[basicstyle=\ttfamily\tiny,firstline=32,lastline=48]{Parser/Com.hs}
	\end{columns}
\end{frame}

\begin{frame}[fragile]
	\frametitle{Command Parsing (\texttt{program}, 2/2)}
	\begin{columns}
		\column{0.45\textwidth}
		\small
		A special note should be given on the while-do construct
		\begin{itemize}
			\small
			\item To deal with iteration, the while construct’s code is read
				into a variable and is repeated after the “end” keyword if the
				boolean condition is true.
			\begin{itemize}
				\footnotesize
				\item Fixed point operator
				\item Function \texttt{repeatWhile}
			\end{itemize}
		\end{itemize}
		\column{0.45\textwidth}
		\lstinputlisting[basicstyle=\ttfamily\tiny,firstline=50,lastline=63]{Parser/Com.hs}
\begin{lstlisting}[basicstyle=\ttfamily\tiny]
repeatWhile :: String -> Parser String
repeatWhile s = P(\env input ->
                  Just (env, "", s ++ input))
\end{lstlisting}
	\end{columns}
\end{frame}


\begin{frame}[fragile]
	\frametitle{The Actual Parsing and Execution}
	\begin{columns}
		\column{0.35\textwidth}
		\small
		A single function is actually exposed
		\begin{itemize}
			\small
		\item eval executes a program in an empty environment
			\begin{itemize}
				\footnotesize
				\item All comments and white space (unless required) are
					stripped from the input program
				\item Functions \texttt{removeComments} and
					\texttt{removeWhitespace}
			\end{itemize}
		\end{itemize}
		\column{0.6\textwidth}
		\lstinputlisting[basicstyle=\ttfamily\tiny,firstline=30,lastline=37]{Parser.hs}
	\end{columns}
\end{frame}

\section{Usage}
\begin{frame}[fragile]
	\frametitle{Usage}
	\begin{columns}
		\column{0.5\textwidth}
		\begin{block}{As a library}
\begin{lstlisting}
$ cd /path/to/src
$ hugs
Hugs> :load Parser
Parser> eval "x := 3"
[x = 3]
\end{lstlisting}
			\end{block}
		\column{0.5\textwidth}
		\begin{block}{As an executable}
\begin{lstlisting}[language=sh]
$ cat file.imp
x := 3;
$ simpli file.imp
[x = 3]
\end{lstlisting}
\begin{lstlisting}[language=sh]
$ echo "x := 3" | simpli
[x = 3]
\end{lstlisting}
\begin{lstlisting}[language=sh]
$ simpli -c "x := 3"
[x = 3]
\end{lstlisting}
		\end{block}
	\end{columns}
\end{frame}

{
	\settocustomtemplate
	\setbeamertemplate{footline}{} 
	\begin{frame}
		\color{white}
		\vfill
		\begin{center}
			\Large
			The End
		\end{center}
		\begin{center}
			\small
			Questions?
		\end{center}
		\vfill
		\begin{center}
			\scriptsize
			Source code:
			\href{https://github.com/espositoandrea/simpli}{%
				\color{mylinkcolor}github.com/espositoandrea/simpli
			}
		\end{center}
	\end{frame}
}
\end{document}
