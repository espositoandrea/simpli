\documentclass{esposito-documentation}
\usepackage[backend=biber,sorting=none]{biblatex}
\usepackage{tikz}
\usepackage{tikz-uml}

\tikzumlset{
	fill component=yellow!0
}

\graphicspath{{images/}}

\title{SIMPLI: the Simple IMP Language Interpreter}
\subtitle{Documentation}
\project{Formal Methods in Computer Science}
\author{Andrea Esposito}
\studentid{735116}
\professor{Prof. Giovanni Pani}
\date{2020/2021}

\makeatletter
\hypersetup{pdfinfo={%
	Author={\@author},
	Title={\@title},
	Subject={\@project},
	Keywords={formal methods, computer science, interpreter, imperative,
		language, haskell}
}}
\makeatother

\lstset{
	inputpath={../src},
	language=haskell,
}

\addbibresource{bibliography.bib}

\begin{document}
\frontmatter
\maketitle
\colophon
\tableofcontents

\mainmatter
\chapter{Introduction}

The following document describes the grammar of a typical imperative language,
``IMP'', and an implementation of the relative interpreter written in the
Haskell language.

\section{The IMP Language}

IMP is a typical imperative programming language. It is, therefore, composed of the classic constructs that all similar languages have:
\begin{description}
	\item[Skip] An instruction that has no side effects. It's only purpose is
		to jump to the next instruction, without any modification to the
		machine's memory.
	\item[Assignment] An instruction that allows the assignment of a value to a
		variable (i.e. a memory location). The assigned value could be both a
		numerical value or a result of an arithmetical computation.
	\item[If-Then-Else] This is the classical conditional control structure.
		Given a boolean condition, if it evaluates to the truth value $\top$
		then the ``\texttt{then}'' section will be executed, otherwise (if the
		condition evaluates to the false value $\bot$) the ``\texttt{else}''
		section will be executed.
	\item[While-Do] An iterative control structure that executes a set of
		instruction as long as a boolean condition evaluates to the truth value
		$\top$.
\end{description}

\subsection{The Language's Grammar}

In this section, the language grammar is presented. It is defined as a Type-3
grammar of a regular language using the Backus-Naur Form. The described grammar
represents the previously defined constructs. For this reason, a set of three
types of expression will be used: $\mathrm{Aexp}$, $\mathrm{Bexp}$ and
$\mathrm{Com}$ (respectively, the set of all arithmetical expressions, of all
boolean expressions and of all commands). For each of this set, a generative
grammar of the expressions is given.

In the following rules, the notation ``a $\mid$ b $\mid$ ... $\mid$ z'' is used
as a short-hand notation to express all the lower case letters from ``a'' to
``z'' (the same applies for capital letters).

\subsubsection{Representation of Numbers}\label{sec:grammar-numbers}

A preliminary grammar to represents integers and natural numbers is here
presented. The IMP language recognizes integers (thus, both positive and
negative natural numbers) expressed in base ten.

\begin{grammar}
	<integer> ::= "-" <natural> | <natural>

	<natural> ::= <digit> | <digit> <natural>

	<digit> ::= "0" | "1" | "2" | "3" | "4" | "5" | "6" | "7" | "8" | "9"
\end{grammar}

\subsubsection{Identifiers and Variables}\label{sec:grammar-id}

A preliminary grammar to represents variables' identifiers is here presented.
The IMP language recognizes alphanumeric identifiers that start with a lower
case letter. The recognized letters are the letters from ``a'' to ``z'', both
in upper and lower case.

\begin{grammar}
	<identifier> ::= <upper> <alphanum>
	\alt <lower> <alphanum>
	\alt <upper> | <lower> 
	
	<alphanum> ::= <upper> <alphanum>
	\alt <lower> <alphanum>
	\alt <natural> <alphanum>
	\alt<upper> | <lower> | <natural>

	<lower> ::= "a" | "b" | ... | "z"

	<upper> ::= "A" | "B" | ... | "Z"
\end{grammar}

\subsubsection{Arithmetical Expressions}\label{sec:grammar-aexp}

A grammar for the expressions in $\mathrm{Aexp}$ is here presented. The
provided operations are sum, difference and product.

\begin{grammar}
	<aexp> ::= <aterm>
	\alt <aterm> "+" <aexp>
	\alt <aterm> "-" <aexp>

	<aterm> ::= <afactor> | <afactor> "*" <aterm>

	<afactor> ::= "(" <aexp> ")"
	\alt <integer> | <identifier>
\end{grammar}

\subsubsection{Boolean Expressions}\label{sec:grammar-bexp}

A grammar for the expressions in $\mathrm{Bexp}$ is here presented. The
provided operations are the logical ``and'', ``or'' and ``not'', alongside with
the comparison operations ``equals to'' and ``less or equal than''.

To simplify the writing of programs in the IMP language, the truth value $\top$
will be represented as ``\texttt{true}'' and the false vale $\bot$ will be
represented as ``\texttt{false}''. Furthermore, the boolean operators $\land,
\vee, \neg$ are represented by the words associated with them (respectively,
``\texttt{and}'', ``\texttt{or}'' and ``\texttt{not}'') and the less or equal
symbol ``$\leq$'' will be represented using ``$<=$'' (as many other languages).

\begin{grammar}
	<bexp> ::= <bterm> "or" <bexp> | <bterm>

	<bterm> ::= <bfactor> "and" <bterm> | <bfactor>

	<bfactor> ::= "true" | "false"
	\alt "!" <bfactor>
	\alt "(" <bexp> ")"
	\alt <bcomparison>

	<bcomparison> ::= <aexp> "=" <aexp> | <aexp> "<=" <aexp>
\end{grammar}

\subsubsection{Imperative Commands}\label{sec:grammar-com}

A grammar for the commands in $\mathrm{Com}$ is here presented. The recognized
commands are the ones presented before.

\begin{grammar}
	<program> ::= <command> | <command> ";" | <command> ";" <program>

	<command> ::= <assignment> | <ifThenElse> | <while> | "skip"

	<assignment> ::= <identifier> ":=" <aexp>

	<ifThenElse> ::= "if" <bexp> "then" <program> "end"
	\alt "if" <bexp> "then" <program> "else" <program> "end"

	<while> ::= "while" <bexp> "do" <program> "end"
\end{grammar}

Given the above grammar, it is clear that the set of keywords of the IMP
language is made of ``\texttt{skip}'', ``\texttt{if}'', ``\texttt{then}'',
``\texttt{else}'', ``\texttt{end}'', ``\texttt{while}'', ``\texttt{do}''. As of
the above grammar, the language is case sensitive.

\section{Additions to the Syntax Given During Lectures}\label{sec:additions}

A few additions are done in this version of the IMP language, that are not
explicitly supported in the syntax showed during lectures but are very common
in commercially-used programming languages. The additions done to IMP are
described in the following paragraphs.

\begin{itemize}
	\item The SIMPLI interpreter supports the insertion of white space to make
		the code more readable for the programmer. This allows to write code
		that spans on multiple lines and to use tabs and/or spaces to indent
		sections of code. All the inserted white space is ignored and discarded
		right before parsing: the only white space kept during the parsing is
		the mandatory white space in the constructs ``if-then-else'' and
		``while-do''.
	\item The SIMPLI interpreter supports the insertion of comments in the
		code. Comments start with an hash (``\texttt{\#}'') and span for the
		entire lines. This is a behaviour inspired by commonly used programming
		languages like the POSIX Shell Command Language \cite{shell-syntax} or
		Python \cite{python-syntax}.
\end{itemize}

\chapter{Design}

The SIMPLI interpreter was defined by following a strategy that is similar to
the one presented in the book ``Programming in Haskell'' \cite{Hutton2017}. So
a set of \emph{Parser}-type monads to check the syntax correctness of the
program and, at the same type, to evaluate expressions (both boolean and
arithmetical) and commands that compose it.

In order to keep track of the memory state of the machine throughout the
computation while keeping the computation in a ``sandbox'' (to protect any
other values in the memory during the computation), the \emph{environment} of
the program is saved and managed. The environment is defined as the set of
variables that the program is managing. Every computation starts with an empty
environment, that is filled during the computation with assignment expressions.
At the end of the computation, the final environment represents the actual
output of the program (IMP does \emph{not} provide any I/O primitives).

\section{The Evaluation Strategy}

For simplicity, the SIMPLI interpreter uses an \emph{eager} evaluation
strategy. In other words, the interpreter evaluates the expressions as soon a
they are encountered. This allows to simplify the implementation of the
interpreter, while actually being a common choice for many other languages,
like C++ \cite{Stroustrup2013}.

Furthermore, the evaluation strategy for boolean expressions should be defined.
In the SIMPLI interpreter, the boolean expressions' evaluation is \emph{short
circuited}: supposing to have two boolean expressions $b_1$ and $b_2$ in
composition using the operators \texttt{and} or \texttt{or}, the interpreter
will not always evaluate both expressions, as defined in the following
paragraphs.
\begin{description}
	\item[And ($\land$)] \hfill \\
		Given the expression $b_1 \land b_2$, the expression $b_1$ is evaluated
		first. If $b_1$ evaluates to the false value $\bot$, then $b_2$ is not
		evaluated (in fact, $\forall b_2: \bot \land b_2 = \bot$).  Otherwise,
		if $b_1$ evaluates to the truth value $\top$, then $b_2$ is evaluated
		and will be the result of the whole evaluation (in fact, $\forall b_2:
		\top\land b_2=b_2$).
	\item[Or ($\vee$)] \hfill \\
		Given the expression $b_1 \vee b_2$, the expression $b_1$ is evaluated
		first. If $b_1$ evaluates to the truth value $\top$, then $b_2$ is not
		evaluated (in fact, $\forall b_2: \top \vee b_2 = \top$).  Otherwise,
		if $b_1$ evaluates to the false value $\bot$, then $b_2$ is evaluated
		and will be the result of the whole evaluation (in fact, $\forall b_2:
		\bot\vee b_2=b_2$).
\end{description}

A short circuited evaluation of boolean expressions allows a faster execution
(potentially less operations are computed) and is as reliable as a non short
circuited evaluation: in the IMP language, in fact, no expression (with the
only exception being the assignment operation) have side effects, thus no
command can depend on the result of the evaluation of the second boolean
expression $b_2$.

\section{The Modules of the Interpreter}

The SIMPLI interpreter is made of four basic modules:
\begin{itemize}
	\item ``\textbf{Main}'' The module that holds the main entry point of the
		interpreter.
	\item ``\textbf{Cli}'' The module that defines the SIMPLI command line
		interface.
	\item ``\textbf{Environment}'' The module that defines the functions to
		manage the computation environment.
	\item ``\textbf{Parser}'' The module that defines the actual parser.
\end{itemize}
The relations between the various modules is represented in
\autoref{fig:uml-design}. The ``Parser'' module can be considered the most
important module and will provide the function ``\texttt{eval}'' that, given a
program, executes it and returns the final environment (starting, as already
stated, from an empty environment).

\begin{figure}[H]
	\centering
	\begin{tikzpicture}
		\umlbasiccomponent[x=5,y=0]{Parser}
		\umlbasiccomponent[x=0, y=-4]{Cli}
		\umlbasiccomponent[x=5, y=-4]{Environment}
		\umlbasiccomponent[x=0, y=0]{Main}

		\umlVHVassemblyconnector[interface=environment]{Parser}{Environment} 
		\umlassemblyconnector[interface=eval]{Main}{Parser}
		\umlVHVassemblyconnector[interface={getInputProgram}]{Main}{Cli} 
	\end{tikzpicture}
	\caption{UML component diagram that represents the basic relations between the modules of the SIMPLI interpreter}
	\label{fig:uml-design}
\end{figure}

\chapter{Implementation}

In the following sections, the implementation of the SIMPLI interpreter will be
discussed and explained. The entire code is also available in the
\autoref{app:source}.
\section{The Main Module}

\lstinputlisting[firstline=1, lastline=1]{Main.hs}

As already stated, the main module contains the definition of the ``main entry
point'' of the SIMPLI interpreter. As it can be seen from
\autoref{fig:uml-design} it uses the interfaces provided by the ``Parser'' and
``Cli'' modules.

\lstinputlisting[firstline=2, lastline=3]{Main.hs}

The main entry point is then defined. 
\lstinputlisting[firstline=5, lastline=8]{Main.hs}

The main function of the program is actually quite simple: its only job is to
get the source code provided through the command line interface and evaluate it
using the Parser module. It then outputs to the standard output the final
environment after the computation.

\section{The Cli Module}

\lstinputlisting[firstline=1, lastline=1]{Cli.hs}

The Cli modules defines and parses the command line interface of the SIMPLI
interpreter. It exports a single function,
\lstinline|getInputProgram :: IO String|, that parses the command line
arguments and returns a String (in the IO monad) containing the source code to
be parsed.

To do that, it uses a set of predefined standard libraries, where the only
exception is the module ``\texttt{Paths_simpli}'', a module automatically
generated by the \emph{Cabal} building system \cite{CabalTeam2020} that holds
the package version.

\lstinputlisting[firstline=2, lastline=8]{Cli.hs}

It then defines two helper functions (that are not exported):
\lstinline|printVersion| and \lstinline|printHelp|, both of type
\lstinline|IO ()|, that respectively print the version of SIMPLI and the help
message that explains the accepted command line arguments and options.

\lstinputlisting[firstline=10, lastline=28]{Cli.hs}

Then, the definition of the accepted options is provided. To the options listed
here, two options are always added: ``\texttt{--help}'' and
``\texttt{--version}'' that, respectively, print the help message and the
interpreter's version.

\lstinputlisting[firstline=30, lastline=30]{Cli.hs}

After the accepted options have been defined, a set of ``default'' options is
defined.

\lstinputlisting[firstline=32, lastline=33]{Cli.hs}

Then, the actual options are defined and mapped to the \lstinline|Options|
record defined before. The function is defined in the IO monad to allow side
effects, in order to simplify the definition of the ``help'' and ``version''
options.

\lstinputlisting[firstline=35, lastline=50]{Cli.hs}

Finally, the actual exported function is defined. It gets the command line arguments using the standard function \lstinline|getArgs| and parses them using the standard function \lstinline|getOpt|. It eventually prints an error message and, if no errors occurred, it applies the default options. It then reads the source code from various sources (depending by the given arguments) in the following order:
\begin{enumerate}
	\item If a command is provided through ``\texttt{--command}'', then the
		provided command is used as source code.
	\item If a file is provided as the first argument, then the provided file's
		content is used as source code.
	\item If no command or file is specified, the source code is read from the
		standard input.
\end{enumerate}
\lstinputlisting[firstline=52, lastline=70]{Cli.hs}

\section{The Environment Module}
\lstinputlisting[firstline=1, lastline=1]{Environment.hs}

The Environment module defines how the environment is managed. It exports all of its content, described in the following paragraphs.

The concept of ``variable'' is defined. A variable is seen as a record having a
name (the identifier of the variable), a type (in this version of the
interpreter, only integers are supported) and a value (the actual value
contained in the variable).

\lstinputlisting[firstline=3, lastline=6]{Environment.hs}

Then, the \lstinline|Variable| type is defined as an instance of the class
\lstinline|Show|, to customize the way a variable is converted to String
(especially when printed as output).

\lstinputlisting[firstline=8, lastline=10]{Environment.hs}

Then, the environment itself is defined as a sequence of variables.

\lstinputlisting[firstline=12, lastline=12]{Environment.hs}

A function to modify an existing environment is then defined. Given an environment and a variable, it returns a new environment that is:
\begin{itemize}
	\item The old environment plus the variable if the variable is not already
		in the environment.
	\item The old environment having the updated variable if the variable was
		already in the old environment (the previous value is then discarded
		after the update).
\end{itemize}
\lstinputlisting[firstline=14, lastline=18]{Environment.hs}

Finally, a function that searches a variable in an environment is provided.
Given an environment and a string, it searches in the environment for a
variable whose name is the given string and returns its value if the variable
is in the environment.

\lstinputlisting[firstline=20, lastline=24]{Environment.hs}

\section{The Parser Module}

To implement the given specification, the ``Parser'' module has been exploded
into multiple sub-modules, described in
\autoref{fig:parser-submodules}.

\begin{figure}[H]
	\centering
	\begin{tikzpicture}
		\begin{umlcomponent}{Parser}
			\umlbasiccomponent[x=0,y=0]{Aexp}
			\umlbasiccomponent[x=4,y=0]{Fundamentals}
			\umlbasiccomponent[x=8,y=0]{Bexp}
			\umlbasiccomponent[x=0,y=-3]{Com}
			\umlbasiccomponent[x=4,y=-3]{Core}
			\umlbasiccomponent[x=8,y=-3]{Environment}
			\umlbasiccomponent[x=4,y=-6]{Readers}
		\end{umlcomponent}
	\end{tikzpicture}
	\caption{UML component diagram that represents the sub-modules of the module ``Parser''}
	\label{fig:parser-submodules}
\end{figure}

\subsection{The Parser Main Module}
\lstinputlisting[firstline=1, lastline=1]{Parser.hs}

The outer Parser module is the one that provides the interface used by the
other modules, so before going into the details of its components, a few words
on it should be given.

First of all, the interfaces of some of its components (\texttt{Com} and
\texttt{Core}) are imported, alongside with the \texttt{Environment} module
(see \autoref{fig:uml-design}) and the standard \texttt{Data.List} module (that
provides useful functions on lists).

\lstinputlisting[firstline=3, lastline=7]{Parser.hs}

Then, a function to remove comments from the source code is provided. It removes, for each line, all characters after an hash symbol (see \autoref{sec:additions}).

\lstinputlisting[firstline=8, lastline=15]{Parser.hs}

After that, a function to remove unnecessary white space is defined. This makes
the implementation of the parsers simpler, because it allows to assume that no
unnecessary characters are in the input string.

\lstinputlisting[firstline=17, lastline=28]{Parser.hs}

Finally, the exported \lstinline|eval| function is provided. Here the actual
execution takes place: it starts from an empty environment and parses a program
(given in input as its only parameter) and it returns the final environment,
modified by the completed execution. It handles errors in the input by simply
raising exceptions through the Haskell's standard \lstinline|error| function.
It should be noted that the function removes all comment and white space from
the given program (in this exact order) using the previously defined functions.

\lstinputlisting[firstline=30, lastline=34]{Parser.hs}

\subsection{The Parser Core Module}
\lstinputlisting[firstline=1, lastline=1]{Parser/Core.hs}

This module defines all the Core variable types and functions used by all other
modules. It starts by defining a Parser type, defining it as an instance of the
classes \lstinline|Functor|, \lstinline|Applicative|, \lstinline|Monad| and
\lstinline|Alternative|.

\lstinputlisting[firstline=5, lastline=5]{Parser/Core.hs}
\lstinputlisting[firstline=11, lastline=15]{Parser/Core.hs}
\lstinputlisting[firstline=17, lastline=24]{Parser/Core.hs}
\lstinputlisting[firstline=26, lastline=30]{Parser/Core.hs}
\lstinputlisting[firstline=32, lastline=39]{Parser/Core.hs}

It also defines a \lstinline|parse| function that is used to actually apply a
parser given an environment.
\lstinputlisting[firstline=7, lastline=8]{Parser/Core.hs}

\subsection{The Parser Fundamentals Module}
\lstinputlisting[firstline=1, lastline=1]{Parser/Fundamentals.hs}

This modules defines some basic parsers that will be used by the parsers for
$\mathrm{Aexp}$, $\mathrm{Bexp}$ and $\mathrm{Com}$. It starts by importing the
core parsers' definitions and some standard libraries.

\lstinputlisting[firstline=3, lastline=6]{Parser/Fundamentals.hs}

And then it defines some useful basic parsers:
\begin{description}
	\item[item] A parser that reads a single character and always succeeds if
		it's applied to a non-empty string.
	\item[failure] A parser that always fails.
	\item[return] A parser that always succeed, returning a value without
		consuming any input.
	\item[satisfies] A parser that reads a character and succeeds if the
		character satisfies a predicate (given as its first parameter).
	\item[symbol] A parser that reads a symbol or a keyword.
\end{description}

\lstinputlisting[firstline=8, lastline=34]{Parser/Fundamentals.hs}

It then defines the parsers for natural numbers by simply applying the grammar
defined in \autoref{sec:grammar-numbers}.

\lstinputlisting[firstline=36, lastline=50]{Parser/Fundamentals.hs}

Finally, it defines the parsers for identifiers by simply applying the grammar
defined in \autoref{sec:grammar-id}.

\lstinputlisting[firstline=52, lastline=93]{Parser/Fundamentals.hs}

\subsection{The Parser Environment Module}
\lstinputlisting[firstline=1, lastline=1]{Parser/Environment.hs}

This module provides the other Parser a simple interface to deal with the
environment management, while still being able of using the functions inside
``\emph{do-blocks}''. To do that, it uses the Environment module and the core
parsers' definitions.

\lstinputlisting[firstline=2, lastline=3]{Parser/Environment.hs}

It then provides a simple function to update an environment given a variable.
This basically is the Parser-monad representation of the function
\lstinline|modifyEnv| provided in the Environment module.

\lstinputlisting[firstline=5, lastline=6]{Parser/Environment.hs}

Finally it provides a simple function to read a variable value from an
environment.  This basically is the Parser-monad representation of the
function \lstinline|searchVariable| provided in the Environment module.

\lstinputlisting[firstline=8, lastline=11]{Parser/Environment.hs}

\subsection{The Parser Aexp Module}
\lstinputlisting[firstline=1, lastline=1]{Parser/Aexp.hs}

This modules provide a parser \lstinline|aexp| for the $\mathrm{Aexp}$
expressions. It uses the other parsers modules, as follows.

\lstinputlisting[firstline=3, lastline=7]{Parser/Aexp.hs}

It then provides the parser for the arithmetical expressions by simply following
the grammar given in \autoref{sec:grammar-aexp}.

\lstinputlisting[firstline=9]{Parser/Aexp.hs}

\subsection{The Parser Bexp Module}
\lstinputlisting[firstline=1, lastline=1]{Parser/Bexp.hs}

This modules provide a parser \lstinline|bexp| for the $\mathrm{Bexp}$
expressions. It uses the other parsers modules, as follows.

\lstinputlisting[firstline=2, lastline=7]{Parser/Bexp.hs}

It then provides the parser for the boolean expressions by simply following the
grammar given in \autoref{sec:grammar-bexp}.

\lstinputlisting[firstline=9]{Parser/Bexp.hs}

\subsection{The Parser Com Module}
\lstinputlisting[firstline=1, lastline=1]{Parser/Com.hs}

This modules provide a parser \lstinline|program| for the $\mathrm{Com}$
commands. It uses the other parsers modules, as follows.

\lstinputlisting[firstline=3, lastline=12]{Parser/Com.hs}

It then provides the parser for the boolean expressions by simply following the
grammar given in \autoref{sec:grammar-com}.

\lstinputlisting[firstline=14]{Parser/Com.hs}

It should be noted that, to deal with the ``while-do'' construct, the parser
simply repeats the while by prepending it to the input string until the
condition is false (function \lstinline|repeatWhile|).

\subsection{The Parser Readers Module}
\lstinputlisting[firstline=1, lastline=1]{Parser/Readers.hs}

This module provides the same exact functions as the \texttt{Aexp},
\texttt{Bexp} and \texttt{Com} modules but changes their definitions in a way
such that no expression is actually executed but the read code is returned.
This is used by other functions in other modules, for example to ``skip''
sections of code.

\lstinputlisting[firstline=2]{Parser/Readers.hs}

\section{Testing of the Implementation}

The provided implementation has been tested on all the commands
($\mathrm{Com}$) by applying the rules of the generative grammar. All the test
cases have been collected inside a unit-test module, \texttt{MainTest.hs}, that
completed the execution successfully on all test cases.

In the \autoref{app:unit-tests} the source code of the unit-test module is
provided.
\chapter{Usage Examples}

In the following sections, various examples of how to use the SIMPLI
interpreter are given.

\section{Reading a Source File}

The SIMPLI interpreter is capable of running a program written in a file. Once
compiled, supposing to have a file ``\texttt{code.imp}'' that contains the
program to be run (written in valid IMP), it can be executed using the
following command.

\begin{lstlisting}[language=sh,numbers=none]
$ cat code.imp # The content of the input file
x := 3; y := x + 4
$ simpli code.imp
[x = 3,y = 7]
\end{lstlisting}

\section{Reading from Standard input}

The SIMPLI interpreter, by default, tries to read the program to be executed
from the standard input. This allows the creation of chains of commands to
produce the source code and, given the fact that the final memory state is
written to the standard output, it is possible to chain commands that use the
output of the given program. To execute SIMPLI and read the source code from
the standard input, simply execute it without any option.

\begin{lstlisting}[language=sh,numbers=none]
$ echo "x := 3; y := x + 4" | simpli
[x = 3,y = 7]
\end{lstlisting}

\section{Interpreting a Command}

The SIMPLI interpreter is capable of running a simple command given as an
argument through its Command Line Interface. Simply provide the command to be
run using the ``\texttt{--command}/\texttt{-c}'' option and SIMPLI will execute
it.

\begin{lstlisting}[language=sh,numbers=none]
$ simpli -c "x := 3; y := x + 4"
[x = 3,y = 7]
\end{lstlisting}

\section{Using It as a Library}

To use SIMPLI as a library (or in Hugs), the only module that needs to be
imported is the module ``\texttt{Parser}'', that contains and exports the
definition of the function \texttt{eval} of type
\lstinline|eval :: String -> Env|.

For example, to use it in an Hugs console, start the Hugs console in the
directory containing the file ``\texttt{Parser.hs}'' and load the file using
\texttt{:load}. Then you can run the interpreter by using the function
\texttt{eval}, where the required argument is the code to be executed.

In the following listing, an example of how to use SIMPLI in Hugs is provided
(supposing Hugs was started in the same directory as ``\texttt{Parser}'').

\begin{lstlisting}[language=sh,numbers=none]
Hugs> :load Parser
Parser> eval "x := 3; y := x + 4"
[x = 3,y = 7]
\end{lstlisting}

\appendix
\chapter{SIMPLI's Source Code}\label{app:source}

In this appendix, all the source code of the various source file is given. All
the source code is available as a Git repository on GitHub as
\href{https://github.com/espositoandrea/simpli}{\texttt{espositoandrea/simpli}}.

SIMPLI is free software: you can redistribute it and/or modify it under the
terms of the GNU General Public License version 3 as published by the Free
Software Foundation. You should have received a copy of the GNU General Public
License along with this program. If not, see
\href{https://www.gnu.org/licenses/gpl-3.0.html}{\texttt{www.gnu.org/licenses/gpl-3.0.html}}.


\section{Main.hs}
\lstinputlisting{Main.hs}

\section{Cli.hs}
\lstinputlisting{Cli.hs}

\section{Environment.hs}
\lstinputlisting{Environment.hs}

\section{Parser}
\subsection{Parser.hs}
\lstinputlisting{Parser.hs}

\subsection{Parser/Core.hs}
\lstinputlisting{Parser/Core.hs}

\subsection{Parser/Fundamentals.hs}
\lstinputlisting{Parser/Fundamentals.hs}

\subsection{Parser/Environment.hs}
\lstinputlisting{Parser/Environment.hs}

\subsection{Parser/Aexp.hs}
\lstinputlisting{Parser/Aexp.hs}

\subsection{Parser/Bexp.hs}
\lstinputlisting{Parser/Bexp.hs}

\subsection{Parser/Com.hs}
\lstinputlisting{Parser/Com.hs}

\subsection{Parser/Readers.hs}
\lstinputlisting{Parser/Readers.hs}

\section{Unit Tests}\label{app:unit-tests}
\lstinputlisting{../test/MainTest.hs}

\backmatter
\printbibliography

\end{document}
